\documentclass{article}
\usepackage[utf8]{inputenc}
\usepackage[spanish]{babel}
\usepackage{listings}
\usepackage{graphicx}
\graphicspath{ {images/} }
\usepackage{cite}

\begin{document}

\begin{titlepage}
    \begin{center}
        \vspace*{1cm}
            
        \Huge
        \textbf{Informe Análisis y Diseño}
            
        \vspace{0.5cm}
        \LARGE
        Parcial 2
            
        \vspace{1.5cm}
            
        \textbf{Manuela Gutiérrez Rodríguez}
        \vspace{0.5cm}
        
        \textbf{Daniela Andrea Gallego Díaz}
            
        \vfill
            
        \vspace{0.8cm}
            
        \Large
        Despartamento de Ingeniería Electrónica y Telecomunicaciones\\
        Universidad de Antioquia\\
        Medellín\\
        21 de Septiembre de 2021
            
    \end{center}
\end{titlepage}

\tableofcontents
\newpage
\section{Análisis del problema}\label{Analisis}



\section{Esquema de tareas} \label{Esquema}

Lista de tareas:

1. En primer lugar diseñar las funciones en qt para el cargue de imagenes y para leer cada uno de los pixeles correspondientes a ésta de acuerdo al model RGB.

2. Realizar un análisis detallado sobre los algoritmos que van a ofrecer la posibilidad de ajustar las imágenes (agrandar o reducir) de acuerdo a las dimesniones de la amtriz de leds.

3. A partir del análisis realizado, implementar un algoritmo en qt que permita la realización y aplicación de dichas funciones.

4. Analizar la matriz que corresponde al modelo RGB de la imagen y como esta debe ser almacenada en un archivo .txt

5. Diseñar el circuito en thinkercad, conectar la cantidad de leds o de tiras de leds para la correcta implementación del algoritmo.

6. Implementar en thinkercad las funciones que a partir del contenido del archivo txt van a permitir mostrar en la matriz de led el color deseado.

7. Realizar el segundo informe de implementación.

8. Realizar un manual de usuario, para el correcto uso del arduino.

9. Realizar el video explicando todo lo realizado.


\section{Algoritmo diseñado}\label{Algoritmo}



\section{Consideraciones}\label{Consideraciones}



\end{document}
